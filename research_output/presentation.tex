\documentclass[aspectratio=169,10pt]{beamer}

\usepackage[spanish]{babel}
\decimalpoint
\usepackage{amsmath}
\usepackage{amssymb}
\usepackage{graphicx}
\usepackage{hyperref}
\usepackage{physics}
\usepackage{xcolor}

% Configuración Beamer
\usetheme{Madrid}
\usecolortheme{default}
\setbeamertemplate{navigation symbols}{}
\setbeamertemplate{footline}[frame number]

% Paleta de colores personalizada
\definecolor{darkblue}{rgb}{0.1, 0.2, 0.5}
\definecolor{lightblue}{rgb}{0.3, 0.5, 0.9}
\setbeamercolor{structure}{fg=darkblue}
\setbeamercolor{alerted text}{fg=red}

\title{Estudio Termodinámico de Partículas Cuánticas en Pozos de Potencial}
\subtitle{Efectos de Temperatura, Estadística y Confinamiento}
\author{Juan Jose Montoya Sánchez \and Juan Pablo Sánchez Arroyave}
\institute{Instituto de Física, Universidad de Antioquia, Medellín, Colombia}
\date{\today}

\begin{document}

% Portada
\frame{\titlepage}

% ====================
% SECCIÓN 1: INTRODUCCIÓN Y MOTIVACIÓN (COMPRIMIDA)
% ====================
\section{Introducción}

\frame{
    \frametitle{Introducción y Motivación}
    \begin{columns}
        \column{0.5\textwidth}
        \textbf{Preguntas Fundamentales:}
        \begin{enumerate}
            \item ¿Comportamiento de sistemas cuánticos confinados a $T$ finita?
            \item ¿Son las correlaciones \textbf{universales} independientemente de geometría?
            \item ¿Cómo conecta microscópico con límite termodinámico?
        \end{enumerate}
        \vspace{0.3cm}
        \textbf{Miranda (2019):} Analizó pozo infinito únicamente
        
        \column{0.5\textwidth}
        \textbf{Nuestro Aporte:}
        \begin{itemize}
            \item Extender a múltiples geometrías (finito, armónico, V)
            \item Sistemas de 2 partículas con efectos de espín
            \item Validar universalidad vs geometría
        \end{itemize}
        \vspace{0.3cm}
        \textbf{Contexto:} Átomos ultrafríos muestran correlaciones robustas en experimentos
    \end{columns}
}

% ====================
% SECCIÓN 2: MARCO TEÓRICO (2 SLIDES COMPRIMIDAS)
% ====================
\section{Marco Teórico}

\frame{
    \frametitle{Fundamentos: Ensamble Canónico y Estadística}
    
    \begin{columns}
        \column{0.5\textwidth}
        \textbf{Matriz Densidad Térmica:}
        \begin{equation*}
        \hat{\rho} = \frac{1}{Z(T)} \sum_n e^{-E_n/k_B T} |n\rangle\langle n|
        \end{equation*}
        
        \textbf{Densidad de Probabilidad:}
        \begin{equation*}
        P_{th}(x, T) = \frac{1}{Z(T)} \sum_{n} |\psi_n(x)|^2 e^{-E_n/k_B T}
        \end{equation*}
        
        \column{0.5\textwidth}
        \textbf{Dos Partículas - Teorema Espín-Estadística:}
        
        Bosones (simétrica):
        \begin{equation*}
        \Psi_B = \frac{1}{\sqrt{2}}[\psi_1\psi_2 + \psi_2\psi_1]
        \end{equation*}
        
        Fermiones (antisimétrica):
        \begin{equation*}
        \Psi_F = \frac{1}{\sqrt{2}}[\psi_1\psi_2 - \psi_2\psi_1]
        \end{equation*}
        
        $\Rightarrow$ Agrupamiento vs Exclusión Pauli
    \end{columns}
}

\frame{
    \frametitle{Efecto del Espín y Métodos Numéricos}
    
    \begin{columns}
        \column{0.5\textwidth}
        \textbf{Modulación por Espín:}
        
        Bosones $s=1$:
        \begin{equation*}
        P^{B,s=1} = \tfrac{8}{9}|\Psi_B|^2 + \tfrac{1}{9}|\Psi_F|^2
        \end{equation*}
        
        Fermiones $s=1/2$:
        \begin{equation*}
        P^{F,s=1/2} = \tfrac{1}{4}|\Psi_B|^2 + \tfrac{3}{4}|\Psi_F|^2
        \end{equation*}
        
        \column{0.5\textwidth}
        \textbf{Métodos Computacionales:}
        \begin{itemize}
            \item \textbf{Numerov:} Pozos finitos (4to orden)
            \item \textbf{Diagonalización:} Potenciales arbitrarios
            \item \textbf{Criterio convergencia:}
        \end{itemize}
        \begin{equation*}
        \max_x |P_{th}^{(N)} - P_{th}^{(N-1)}| < 10^{-4}
        \end{equation*}
    \end{columns}
}

% ====================
% SECCIÓN 3: RESULTADOS - TERMALIZACIÓN
% ====================
\section{Resultados: Termalización}

\frame{
    \frametitle{Densidades Térmicas en Diferentes Confinamientos}
    \begin{columns}
        \column{0.65\textwidth}
        \includegraphics[width=\linewidth,height=0.75\textheight,keepaspectratio]{figuresnew/1.png}
        
        \column{0.33\textwidth}
        \small Temperatura homogeneiza:
        \begin{itemize}
            \item $t=1$: Estructura nodal
            \item $t=50$: Distribución uniforme
        \end{itemize}
        \vspace{0.3cm}
        Tunelamiento emerge en $V_0=10$
    \end{columns}
}

\frame{
    \frametitle{Capa Límite Térmica y Convergencia}
    \begin{columns}
        \column{0.48\linewidth}
        \includegraphics[width=\linewidth,height=0.65\textheight,keepaspectratio]{figuresnew/2.png}
        \column{0.48\linewidth}
        \includegraphics[width=\linewidth,height=0.65\textheight,keepaspectratio]{figuresnew/3.png}
    \end{columns}
    \small Izq: Extensión de capa límite (valores negativos = tunelamiento). Der: Estados requeridos crece con $T$, pero depende del potencial.
}

% ====================
% SECCIÓN 4: CORRELACIONES DE DOS PARTÍCULAS
% ====================
\section{Correlaciones}

\frame{
    \frametitle{Bosones vs Fermiones en Pozos Cuadrados}
    \begin{columns}
        \column{0.48\linewidth}
        \includegraphics[width=\linewidth,height=0.65\textheight,keepaspectratio]{figuresnew/4.png}
        \column{0.48\linewidth}
        \includegraphics[width=\linewidth,height=0.65\textheight,keepaspectratio]{figuresnew/5.png}
    \end{columns}
    \small \textbf{Izq:} Bosones muestran máximo en diagonal. \textbf{Der:} Fermiones muestran exclusión en diagonal.
}

\frame{
    \frametitle{Promedios Térmicos y Efecto del Espín}
    \begin{columns}
        \column{0.48\linewidth}
        \includegraphics[width=\linewidth,height=0.65\textheight,keepaspectratio]{figuresnew/6.png}
        \column{0.48\linewidth}
        \includegraphics[width=\linewidth,height=0.65\textheight,keepaspectratio]{figuresnew/7.png}
    \end{columns}
    \small \textbf{Izq:} Correlaciones persisten a $t=10$ (suavizadas). \textbf{Der:} Espín modula dramáticamente: mezclas 8/9+1/9 y 1/4+3/4.
}

% ====================
% SECCIÓN 5: UNIVERSALIDAD
% ====================
\section{Universalidad}

\frame{
    \frametitle{Potenciales Suaves: Bosones}
    \begin{columns}
        \column{0.65\textwidth}
        \includegraphics[width=\linewidth,height=0.75\textheight,keepaspectratio]{figuresnew/8.png}
        
        \column{0.33\textwidth}
        \small 
        \textbf{Geometrías:}
        \begin{itemize}
            \item Armónico truncado (Hermite-Gaussianas)
            \item Pozo en V (Airy)
        \end{itemize}
        \vspace{0.3cm}
        Ambos muestran máximo en diagonal.
        \vspace{0.3cm}
        \textbf{Correlación es universal}
    \end{columns}
}

\frame{
    \frametitle{Potenciales Suaves: Fermiones y Promedios}
    \begin{columns}
        \column{0.48\linewidth}
        \includegraphics[width=\linewidth,height=0.65\textheight,keepaspectratio]{figuresnew/9.png}
        \column{0.48\linewidth}
        \includegraphics[width=\linewidth,height=0.65\textheight,keepaspectratio]{figuresnew/10.png}
    \end{columns}
    \small Fermiones (izq) y promedios térmicos (der) en potenciales suaves. Estructuras \textbf{idénticas} a pozos cuadrados: universalidad independiente de geometría.
}

\frame{
    \frametitle{Efecto del Espín: Potenciales Suaves}
    \begin{columns}
        \column{0.65\textwidth}
        \includegraphics[width=\linewidth,height=0.75\textheight,keepaspectratio]{figuresnew/11.png}
        
        \column{0.33\textwidth}
        \small 
        Bosones vs fermiones con/sin espín en potenciales suaves.
        \vspace{0.4cm}
        
        \textbf{Conclusión:}
        \vspace{0.2cm}
        
        Mezclas espín-estadística son \textbf{universales} a través de todas las geometrías
    \end{columns}
}

% ====================
% SECCIÓN 6: VALIDACIÓN
% ====================
\section{Validación}

\frame{
    \frametitle{Densidad de Estados: Límite Termodinámico}
    \begin{columns}
        \column{0.65\textwidth}
        \includegraphics[width=\linewidth,height=0.75\textheight,keepaspectratio]{figuresnew/12.png}
        
        \column{0.33\textwidth}
        \small 
        \textbf{Predicción teórica:}
        \vspace{0.2cm}
        
        $g(E) \propto E^{-1/2}$
        \vspace{0.4cm}
        
        \textbf{Convergencia:}
        \vspace{0.2cm}
        
        Espectro discreto → continuo conforme $L \to \infty$, $V_0 \to \infty$
        \vspace{0.4cm}
        
        Propiedades macroscópicas emergen correctamente
    \end{columns}
}

% ====================
% SECCIÓN 7: CONCLUSIONES (COMPRIMIDA)
% ====================
\section{Conclusiones}

\frame{
    \frametitle{Síntesis y Conclusiones}
    
    \begin{enumerate}
        \item \textbf{Termalización Homogeneizadora:} Temperatura es agente universal que borra estructura nodal. Detalles microscópicos importan cuantitativamente para simulaciones.
        
        \vspace{0.2cm}
        \item \textbf{Robustez de Correlaciones:} Agrupamiento bosónico y exclusión fermiónica son \textbf{independientes de geometría}. Efecto espín es fundamental y consistente. Validado experimentalmente en átomos ultrafríos.
        
        \vspace{0.2cm}
        \item \textbf{Convergencia Termodinámica:} Espectro discreto → límite continuo. Propiedades macroscópicas emergen de microscópicas correctamente (DOS universal).
        
        \vspace{0.2cm}
        \item \textbf{Marco Conceptual:} Integra mecánica cuántica fundamental + física estadística + fenomenología de átomos ultrafríos. Aplicable a sistemas nanométricos y dispositivos cuánticos.
    \end{enumerate}
}

% Slide de cierre
\frame{
    \begin{center}
        {\Large \textbf{Gracias}}
        \vspace{1cm}
        
        \textbf{Preguntas}
    \end{center}
}

\end{document}
